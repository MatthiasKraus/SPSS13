% !TEX encoding = UTF-8 Unicode
% Vorlage zur Generierung von Uebungsloesungsblaettern.



\thispagestyle{empty} %Seitennummerierung 1 ausblenden


\begin{minipage}[l]{0.6\textwidth}
{\large \today}\\

{\large Szenarionummer \rule{1cm}{0.4pt}}\\

{\large Testmanager \rule{4cm}{0.4pt}}\\

\end{minipage}
\begin{minipage}{0.4\textwidth}
\includegraphics*[width=6cm]{Logo.png}
\end{minipage}\\


\vspace*{1cm} 
\begin{center}
\textbf{{\LARGE Software Testszenario}}\\
\end{center}

\begin{center}
{\large ISGCI - Information System on Graph Classes and their Inclusions}\\
\end{center}


$ $\\

\fbox{
	\begin{minipage}{1.0\textwidth}
		\begin{center}
	\textit{Szenariobeschreibung}
	\end{center}
		Auswählen eines Knotens und einblenden seiner Superklassen mithilfe des Kontextmenüs.\\
	\end{minipage}
	}

	$ $\\

\begin{tabular}{|c|l|}
\hline
\rule{0.00\textwidth}{0.5cm} \textbf{Schrittnummer} & \textbf{Ereignis/Aktion} \hspace{0.55\textwidth} \\[0.25cm]
\hline 
\rule{0.00\textwidth}{0.5cm} 00 & Vorbedingung: Ein Graph wurde gezeichnet. \\[0.25cm]
\hline
 \rule{0.00\textwidth}{0.5cm} 01 & Der Benutzer klickt mit der rechten Maustaste auf einen Knoten. \\[0.25cm]
\hline 
 \rule{0.00\textwidth}{0.5cm} 02 & Es erscheint ein Kontextmenü. \\[0.25cm]
\hline 
 \rule{0.00\textwidth}{0.5cm} 03 & Der Benutzer klickt auf Show Superclasses. \\[0.25cm]
\hline
\end{tabular} \ \\

\fbox{
	\begin{minipage}{1.0\textwidth}
	\textit{Erwünschte Reaktion} \\
		Superklassen zum ausgewählten Knoten sind eingeblendet und Kontextmenü ist wieder geschlossen. \\
	\end{minipage}
	} \ \\
	
	\fbox{
	\begin{minipage}{1.0\textwidth}
	\textit{Tatsächliche Reaktion}
		\hfill\vspace{1.0cm}
	\end{minipage}
	} \ \\
	
	\fbox{
	\begin{minipage}{1.0\textwidth}
	\textit{Wertung}
		\hfill\vspace{1.0cm}
	\end{minipage}
	}


