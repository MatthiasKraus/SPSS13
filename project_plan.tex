\documentclass[11pt,a4paper]{article}
\usepackage[T1]{fontenc}
\usepackage{pgfgantt} % Timeline
\usetikzlibrary{decorations.pathmorphing} %Zigzag - Timeline
\usepackage{lmodern}
\usepackage[utf8]{inputenc}
\usepackage{hyperref}

\newganttlinktype{zigzag}{%
\ganttsetstartanchor{on right=1}%
\ganttsetendanchor{on left=0}%
\draw [decoration=zigzag, decorate, thick, cyan]
(\xLeft, \yUpper) --
(\xRight, \yLower);%
}
\begin{document}

\title{Project Plan}
\author{Graph Maga}
\date{\today}
\maketitle

\tableofcontents

%Introduction, High Level overview

\clearpage

\section{Introduction}
Ernst de Ridder is CEO of \url{www.graphclasses.org} and owner of ISGCI, a graph drawing application using the Java library "JgraphT". This application was developed 15 years ago and has been updated to work with modern software. ISGCI is currently the biggest online information system about graph classes, as well as a useful library for many users. It is an open source application, that lets you create graph drawings for various purposes. Our projects goal is to improve the current functionality and implement additional features, using different libraries, in order to make the website more attractive and reduce the maintenance cost of the system.


\section{Drawing library}
The different library mentioned, is called "JGraphX", which is a Swing graph visualization library for the use in desktop environments.

\section{Members and Responsibilities}

{\small 
\hspace*{0.6cm}\textbf{\textit{Project Manager: Leonard Krämer}} \\
{\footnotesize The Project Manager is responsible for the product being delivered in time. He organizes all tasks and keeps track of changes. Furthermore he leads all meetings and takes care of the risk management. For the customer he is the main contact person regarding the product.  }\\

\textbf{\textit{Document Manager: Matthias Miller}}\\
{\footnotesize The Document Manager is in charge of the product documentation, as well as the development documentation. He rather specifies the boundaries of a good documentation than writing it on his own. His tasks also cover keeping track of changes and taking care of a coherent and complete documentation.} \\

\textbf{\textit{Test Manager: Fabian Vollmer}}\\
{\footnotesize The Test Manager is responsible for a well tested and working Software. He ensures that the software performs and works closely with the Specification Manager.  }}\\

\textbf{\textit{Version- \& File Manager: Rebecca Kehlbeck}} \\
{\footnotesize The Version \& File Manager is in charge of proper backups and a well-conceived file structure. Another important task is to make the team use consistent files and versioning software. }\\

\textbf{\textit{Specification Manager: Matthias Kraus}} \\
{\footnotesize The Specification Manager is responsible for the elicitation of all requirements and the creation of the SRS \slash SDD. He defines the standard of these documents and ensures their correctness throughout the creation.} \\

\textbf{\textit{Media Manager: Nico Siebeck}} \\
{\footnotesize The Media Manager is assigned to the creation of protocols and other presentable documents. Other various tasks will depend on the use of other resources.}\\

\textbf{\textit{Vice Project Manager: Philipp Hilpert }}\\
{\footnotesize The Vice Project Manager represents the Project Manager when unavailable and helps him to keep track of everything. Provides ideas and helps all other managers with their tasks.  } \\

All team members will contribute to the coding and documentation.


\section{Development methodology}

Through the use of the V-Model approach, we will ensure a good documentation and a well defined process.

\section{Project Organization}

This section provides information about the communication used by our team and shows how we will use common tools to communicate and keep track of changes. It includes an overview of tools and forms of communication. A table of all contact information can be found online \footnote[1]{goo.gl/tVUzp}.

\subsection{Communication Management Plan}
\begin{itemize}
\item[Meetings:] We will meet weekly and discuss our success, issues and find solutions. The Project Manager leads the discussion while the Media Manager writes Protocol. Decisions are made by the whole Team. If the Project Manager is unavailable he will be represented by the Vice Project Manager. Meeting Minutes will be sent to the Customer after each meeting.
\item[Skype:] For quick help and shorter discussions we use a Skype Group which can be used to inform other Team Members about important events. This should be avoided if not important. For emergencies and to fix appointments there is a Whatsapp Group. 
\item[E-Mail:] E-Mails are assigned a meaningful tag. For example [SWP] - Team Meeting. Whenever team members assign tasks to each other or transfer responsibilities the Project \& or Vice Project Manager needs to be put in CC to assure transparency. Each member must check email daily.
\item[Google-Group:] All found information that is not well maintainable in a chat or email can be stored in a Google-Group. It can be used to save tutorials and other links to websites\slash projects or files, important to the project.
\item[Informal Communications:] Any questions \& issues that arise when team members communicate with each other have to be forwarded to the Project Manager so that it can be acted accordingly.  
\end{itemize}

\subsection{Group Rules}
\begin{enumerate}
\item If a member is not able to attend to a meeting because of sickness or any other reason, he has to notify the Project Manager or Vice Project Manager via E-Mail of his absence as soon as possible, when possible one or two days before.
\item Respect for other members.
\item Every member shall give his best.
\item The amount of work shall be divided evenly among all team members.
\item Meetings must be attended on time. 
\item All tasks have to be completed at the stated time. If a member can't complete a task till the stated time, he shall tell the V. PM. or PM. and ask for help or more time.
\item Every member shall check his E-Mail Account daily, and respond to messages as soon as possible to guarantee a good communication.
\item We will meet weekly. If necessary more often. 
\item Weekends are not working hours.
\item Use version control. 
\item Tag email appropriately.
\end{enumerate}

\newpage
\subsection{Milestones and Project Plan}
{\small\begin{enumerate}
  \item Project Plan 30.04
  \item 1st Prototype
  \item SRS 21.05
  \item SDD 04.06 
  \item Alpha version 28.05.
  \item Beta version 12.06.
  \item Release Candidate 02.07.
  \item Final Demo and release 16.07 
\end{enumerate}}
\begin{center}
\hspace*{-2.5cm}{
\begin{ganttchart}[ hgrid,  vgrid, inline, x unit=1.2cm, ,
y unit chart=0.9cm, bar/.style={fill=green},
%
incomplete/.style={fill=yellow},
today = 1
%bar/.style={fill=yellow!60, draw=none}
]{12} % dashed, 
\gantttitle[title/.style={fill=gray!80}]{\textbf{Project Plan}}{12}\\
\gantttitlelist[title/.style={fill=gray!15}]{30.04.,7.05.,14.05,21.05,28.05,04.06.,11.06,18.06,25.06,02.07., 09.07.,16.07.}{1} \\
\ganttbar[progress=100, progress label text={}]{\hspace*{1cm}{\scriptsize Project Plan}}{1}{1}  \\
\ganttmilestone{\footnotesize Plan}{1}\\ % [bar label inline anchor/.style=above] (right, left)
\ganttbar[progress=30, progress label text={}]{\footnotesize Req. Elicitation\hspace*{0.5cm}}{2}{2} \\
\ganttmilestone{\footnotesize 1st Prototype}{2} \\
\ganttbar[progress=10, progress label text={}]{{\footnotesize Software Specification\hspace*{0.5cm}}}{2}{4} 
\ganttmilestone{\footnotesize SRS}{4} \\
\ganttbar[progress=00, progress label text={}]{{\footnotesize System Design}}{3}{6} 
\ganttmilestone{\footnotesize SDD}{6}\\
\ganttbar[progress=00, progress label text={}]{{\footnotesize Implementation}}{4}{8} \\


\ganttmilestone{\footnotesize Alpha Version}{5} 
\ganttmilestone{\footnotesize Beta Version}{8} \\

\ganttbar[progress=00, progress label text={}]{{\footnotesize Module Testing}}{6}{8} \\

%\ganttbar{{\footnotesize Prototyp I}}{10}{14} \\
%\ganttbar{{\footnotesize Fertigstellung}}{13}{16} \\
\ganttbar[progress=00, progress label text={}]{{\footnotesize System Testing}}{9}{10} \\
\ganttmilestone{\footnotesize Release Candidate}{10}\\
\ganttbar[progress=00, progress label text={}]{\footnotesize Acceptance testing}{11}{12}\\
\ganttbar[progress=00, progress label text={}]{\hspace*{-0.8cm}{\scriptsize Final Demo}}{12}{12} \\
\ganttmilestone{\footnotesize Release}{12}
%\ganttlink[link type=zigzag]{elem6}{elem7}

\end{ganttchart}}


\end{center}


\subsection{Progress Tracking and Issue Tracking}

To track progress of our project we will use the meetings and Github. At the beginning of every meeting we will check if all tasks have been completed by the team members, if not, we will discuss why the task hadn't been finished, assign a new deadline or help to resolve the issue as fast as possible. Github will help to track issues, when we are in implementation stage. It will help to provide an overview of which features have been implemented and which milestones accomplished. Optionally we will push the SRS and SDD to Github to keep track of sections to be added and content missing to complete the documentation. In addition the progress will be shown in a diagram as above.

Issues will be added to Github and marked as resolved when finished. 




\end{document}
