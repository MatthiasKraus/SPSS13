\documentclass[11pt,a4paper]{article}

\usepackage[en,break]{ukon-infie}
\usepackage[T1]{fontenc}
\usepackage{lmodern}
\usepackage[utf8]{inputenc}
%\usepackage{pgfgantt} % Timeline 

\usepackage{amsmath}
\usepackage{graphicx}
\usepackage{wasysym}
\usepackage{courier}
\usepackage{textcomp}

\begin{document}
% HEAD |-->
\thispagestyle{empty} %Seitennummerierung 1 ausblenden
\vspace*{5cm} 
\begin{center}
{\huge Software Requirements Specification}\\
{\large Version 1.0}\\
{\large 21. Mai 2013}\\
\end{center}

\begin{center}
{\large ISGCI - Information System on Graph Classes and their Inclusions}\\
{\large Team $Graph$ $Maga$}
\end{center}
% HEAD -->|




%\title{{\LARGE \textbf{Software Requirements Specification}}\\ \textbf{{\large Document}}}
	%\begin{center}
	%\textbf{{\huge Software Requirements Specification}}
	%\end{center}
%\author{\textbf{Graph Maga}}
%\date{\today}
%\maketitle
% \clearpage
 
 
 
 
 \newpage
 
 % Inhaltsverzeichnis |-->
 \renewcommand{\contentsname}{Inhaltsverzeichnis}
 \tableofcontents % Definiert durch Sections/Subsections
 % Inhaltsverzeichnis -->|
 
 \newpage
 
 
 
 
%{\large \textbf{Kommentar (Matze): }

%\textit{Das Dokument ist jetzt nach dem IEEE 830-1998 SRS Standard gerichtet (SE Slides 2-180) und bereits ein wenig modifiziert. Sollte jemand Verbesserungsvorschläge haben, dann kann er sie selbst ändern und/oder mir Bescheid sagen und ich erledige das dann, falls noch nicht geschehen.
%Abgesehen davon sollte man noch überlegen, in welchen Bereich des Dokuments "System Models", "System Evolution", "System Architecture" (Sommerville, Slides 2-178) passen würde.}
%
%\textit{Ich habe jetzt bei jeder Überschrift, sowohl die deutsche als auch englische Überschrift reingeschrieben, die annähernd das selbe beschreiben, vllt kann der ein oder andere mit der anderen Überschrift mehr anfangen, als der andere. Am Ende würde ich sagen entscheiden wir uns für eine Sprache, und da wir das Dokument auf deutsch schreiben wollen, bin ich dafür, dass wir dementsprechend auch die Überschriften deutsch machen. Wenn das nichts ausmachen gern auch englisch, aber dann würde ich das mit Stephan abklären.}
%
%\textit{Die geschweiften Klammern in den einzelnen Bereichen sollen darauf hindeuten, dass dies durch den späteren Inhalt ersetzt werden soll. In den Klammern hab ich versucht, falls möglich zu beschreiben, wie der Inhalt ganz grob aussehen könnte.}
%
%\textit{Ich hoffe, dass aber die Struktur ansonsten allen zusagt.. wenn soweit ist, dann haut diesen Kommentar einfach aus dem Dokument oder kommentiert es im Source-Code aus.. dann kann man nochmal nachschauen, wenn man möchte. }


\section{Einleitung / Introduction} % 1. 
	\{definiere erwartete Leserschaft des Dokuments \& Version History.\\
    Grund für die Erstellung der neuen Version, bzw. Zusammenfassung der Veränderungen\}
 	\subsection{Zweck (des Dokuments) / Purpose} %1.1
	\{Wieso braucht man das System. Kurze Beschreibung der System Funktionen \& wie es mit anderen Systemen arbeiten soll.\\
	(wie passt das System in gesamte Unternehmen || stragische Ziele der Organisation d. Software)\}
  	\subsection{Umfang (des Softwareprodukts) / Scope} %1.2
   \{...\}
  	\subsection{Erläuterungen zu Begriffen und / oder Abkürzungen (Glossar) / Definitions, acronyms and abbreviations} %1.3
	\{        Definiere technische Begriffe, die im Dokument verwendet werden und nicht "klar" sind.
	        Keine Annahmen über die Erfahrungen und Fachkenntnis des Lesers machen.\}
  	\subsection{Verweise auf sonstige Ressourcen oder Quellen / References} %1.4
  	\{Hyperlinks? http://www.graphclasses.org/\\
          ISGCI, JGraphX als Grafikbibliothek, JGraphT\}
  	\subsection{Übersicht/Überblick (Wie ist das Dokument aufgebaut?) / Overview} %1.5
  	\{Ist dieser Abschnitt notwendig?!.. zur Not einfach kippen\}
  	
\newpage
\section{Allgemeine Beschreibung} % (des Softwareprodukts) / Overall Description} %2.
  	\subsection{Produktperspektive} % (zu anderen Softwareprodukten) / Product Perspective} %2.1
	\{Beschreibung, dass unseres Produkt als "Schnittstelle" zwischen JGraphX und ISGCI fungieren wird\\
	und wie diese miteinander interagieren, bzw. wie darauf aufgebaut wird.\}
  	\subsection{Produktfunktionen} % (eine Zusammenfassung und Übersicht) / Product Functions}  %2.2
  	
	\begin{description}[leftmargin=.9cm]
\item[1] Das System soll einen Graphen zeichnen können, welcher Relationen zwischen Graphklassen zeigt.
\item[2] Das System soll den erzeugten Graphen exportieren können.
\item[3] Das System soll dazu in der Lage sein eine neue Fensterinstanz zu öffnen.
\item[4] Das System soll die Möglichkeit bieten im gezeichneten Graphen eine Graphklasse zu suchen. 
\item[5] Das System soll dazu in der Lage sein die Namensgebung der Graphklassen innerhalb des erzeugten Graphen nach vorgegebenen Mustern zu ändern.
\item[6] Das System soll Fehlerhafte Inklusionen markieren können.
\item[7] Das System soll die Graphdatenbank und die Eigenschaften der Graphklassen anzeigen können. 
\item[8] Das System soll die Möglichkeit bieten nach Relationen zwischen Graphklassen zu suchen.
\item[9] Das System soll die Möglichkeit bieten abhängig eines Graph-Problems die entsprechenden Graphklassen zu zeichnen.
\item[10] Das System soll die Möglichkeit bieten Graph-Probleme im erzeugten Graphen anzuzeigen.
\item[11] Das System soll einen Hyper-Link zu "small graphs" besitzen.
\item[12] Das System soll einen Hyper-Link zur Hilfe Website von ISGCI besitzen.
\item[13] Das System soll eine About Funktion besitzen.
\item[14] Das System soll die Möglichkeit bieten einen Knoten im erzeugten Graphen zu markieren.
\item[15] Das System soll die Möglichkeit bieten einen Knoten im erzeugten Graphen zu verschieben.
\item[16] Das System soll die Möglichkeit bieten über ein Kontextmenü zu einer Graphklasse im erzeugten Graphen weitere Informationen zu dieser zu erhalten.
\item[17] Das System soll die Möglichkeit bieten im erzeugten Graphen zu zoomen.
\item[18] Das System soll einen Tooltip anzeigen wenn der Mauszeiger auf einem Knoten im erzeugten Graphen steht.
\item[19] Das System soll, wenn ein Graph gezeichnet wird, die für die Zeichnung ausgewählte Graphklasse im Bild zentriert darstellen.
\item[20] Das System soll die Möglichkeit bieten durch das gedrückt halten der Linken Maustaste und ziehen den Bildausschnitt zu verschieben.
\item[21] Das System soll eine Funktion beinhalten mithilfe derer man Nachbarknoten beziehungsweise Super- und Subklassen der Gewählten Graphklasse ein-/ausblenden kann.

\end{description}
     
    \subsection{Benutzermerkmale} % (Informationen zu erwarteten Nutzern, z.B. Bildung, Erfahrung, Sachkenntnis) / User characteristics} %2.3
	Die Zielgruppe des fertigen umgesetzten Systems sind in erster Linie Personen, die sich mit den grundlegenden Eigenschaften bereits auskennen und dementsprechend weitere Forschungen in Bereich Graphen und deren Eigenschaften anstellen wollen. Daher werden viele Grundlagen nicht näher erläutert. Vielmehr sollen Personen, die sich mit Zusammenhängen verschiedener Graph-Klassen näher beschäftigen, durch das fertige System einfach Zugang zu Informationen verschaffen können, ohne dass sie weitere Hilfe zur Nutzung des gesamten Systems benötigen.
	Letztendlich dient das System also zu Forschung und zum besseren Verständnis von Graphen im Studium oder Beruf .
	%\{Welche Kenntnisse haben die potentiellen Nutzer des Systems, bzw. welche Personengruppen werden das System nutzen.\}    
	\subsection{Einschränkungen} % (für den Entwickler) / Constraints} %2.4
	Die Implementation des Systems ist Java-basiert und die aktuelle Inkompatibilität mit aktuellen Java-Versionen erfordert eine Erneuerung des Systems. Dies führt zu verschiedenen Beschränkungen, die beachtet werden müssen.\\
	Die Umsetzung findet für Java-Versionen ab Java 1.6 statt. Deshalb kann bei niedrigeren Java-Versionen keine Funktionalität gewährleistet werden.\\
	Die graphische Verbesserung des bestehenden Systems wird insbesondere über folgende Bibliotheken umgesetzt:
	JGraphX wird als neue Zeichenbibliothek fungieren.\\
	Die Umsetzung findet statt, indem die Funktionalität von JGraphT auf JGraphX übertragen wird und anschließend ISGCI an JGraphX angebunden wird. So wird JGraphT letztendlich nicht zur Ausführung der Software verwendet.\\ %Leo: Wie ist das gemeint? JgraphT braucht man also nicht?
	Dementsprechend ist ausdrücklich gesagt, dass alle Beschränkungen, die für die genannten Bibliotheken gelten, insbesondere für die fertige Implementation gelten.\\
	Die Software wird nach Umsetzung auf jedem System ausführbar sein, welches die angegebenen Voraussetzungen erfüllt (Java 1.6 +).\\
	Die Software wird als Open-Source-Project %Englisch nötig?
gehandhabt. Daher wird jede Dokumentation, die zur Erweiterung des Source-Codes stattfindet, in englischer Sprache formuliert, sodass unsere Implementation zur weiteren Nutzung/Forschung beitragen kann. 
	%\{zeitliche Einschränkungen (Enddatum der Abgabe), Plattformen, Version, vorgegebene Bibliotheken, Implementationsaufwand, kundenorientiert!!!\}
%	\subsection{Annahmen und Abhängigkeiten (nicht Realisierbares und auf spätere Versionen verschobene Eigenschaften) / Assumptions and dependencies} %2.5 
%	\{Brain-Storming :-)\}
	\subsection{Anteil der Anforderungen} %im Gesamten / Apportioning of requirements} %2.6
	Im nächsten Kapitel beschriebene funktionale Anforderungen müssen umgesetzt werden, damit das System/deren Erweiterung die entsprechende Funktionalität hat, damit ISGCI an JGraphX gebunden werden kann und alle bisherigen und hinzukommende Funktionen (außer evtl. Widersprüche) ausgeführt werden können.
\newpage
\section{Spezifische Anforderungen} % (im Gegensatz zu 2.) / Specific requirements} %3.
	\subsection{Funktionale Anforderungen (Stark abhängig von der Art des Softwareprodukts)} %3.1
	
	Die bisher vorhandenen funktionalen Anforderungen werden, soweit nicht abweichend angegeben, übernommen.

\colorbox{red}{Möglicherweise nochmal überarbeiten da requirements ja schon am Freitag rausgeschickt}

\begin{tabular}{|c|c|p{10cm}|}
\hline 
 & Name & Beschreibung \\ 
\hline 
1 & Knoten Verschieben & Die Funktion wird dahingehend abgeändert dass ein Knoten vor dem Verschieben markiert werden muss. \\ 
\hline 
2 & Zoom (Mausrad) & Durch das Scrollen mit dem Mausrad wird das Bild gezoomt, Mausrad nach oben entspricht hinein zoomen, Mausrad nach unten entspricht heraus zoomen. \\ 
\hline 
3 & Zoom (Menu-Bar) & Dem Menu-Item View werden die Funktionen "Hinein zoomen" und "Heraus zoomen" hinzugefügt. \\ 
\hline 
4 & Tooltip & Verweilt die Maus auf einem Knoten wird ein Tooltip angezeigt. \\ 
\hline 
5 & Tooltip Inhalt & Der angezeigte Tooltip entspricht den Informationen zum jeweiligen Knoten, beziehungsweise einer verkürzten Form der Informationen. \\ 
\hline 
6 & Zentrierung & Wird ein neuer Graph gezeichnet ist der zum Zeichnen ausgewählte Knoten im Mittelpunkt des Bildes. \\ 
\hline 
7 & Scrolling & Klickt man in den freien Raum und hält die linke Maustaste gedrückt so kann man durch das Bewegen der Maus den angezeigten Bildausschnitt verschieben.  \\ 
\hline 
8 & Kontextmenü & Man kann Knoten mit der rechten Maustaste anklicken um ein Kontextmenü zu öffnen. \\ 
\hline 
9 & Kontextmenü Superklassen & Das Kontextmenü enthält die Funktion, die dem Knoten zugehörigen Superklassen ein- oder auszublenden \\ 
\hline 
10 & Kontextmenü Subklassen & Das Kontextmenü enthält die Funktion, die dem Knoten zugehörigen Subklassen ein- oder auszublenden \\ 
\hline 
11 & Kontextmenü Nachbarn & Das Kontextmenü enthält die Funktion, die dem Knoten zugehörigen Nachbarknoten ein- oder auszublenden.  \\ 
\hline 
\end{tabular} 

	
	\subsection{Nicht-funktionale Anforderungen} %3.2     
		\begin{itemize}
		\item Technische Grundlage des Softwareprojekts bietet die Programmiersprache Java.
		\item Kompatibilität: Die Software ISGCI wird Java-Runtime-Umgebungen mit Java 1.6 und Nachfolger vollständig unterstützen, solange diese rückwärtskompatibel zu Java 1.6 sind. Für ältere Versionen wird die Lauffähigkeit nicht garantiert.
		\item Zur Umsetzung graphischer Anwendungen wird die Grafikbibliothek JGraphX verwendet.
		\item Das Projekt wird als OpenSource veröffentlicht.
		\item Das Benutzerinterface soll interaktiver und userfreundlicher sein als bei der Vorgängerversion (Umsetzung: vgl. Functional Requirements).
		\item Dokumentationen werden auf Englisch verfasst, um sich dem OpenSource Standard und der restlichen Dokumentation anzupassen.
		\item Zeichnungen im ISGCI sollen visuell attraktiver dargestellt werden.
		\item Datenbank-basiertes zeichnen: Flexible Einspeisung von Datensätzen anstatt statischer Einbindung führt zu einer bessern Wartbarkeit.
		\end{itemize}
		
	\subsection{Externe Schnittstellen - Anforderungen} %3.3
	\begin{itemize}
	\item[] \textit{Benutzerschnittstellen:}\\
	Der Benutzer benötigt zum vollständigen Nutzen des Systems verschiedene Schnittstellen.\\
	Er benötigt Peripheriegeräte (Maus, Tastatur) und deren Treiberunterstützung entsprechender Hersteller, damit die Funktionalitäten des Systems nutzbar sind.\\
	Der Benutzer benötigt eine grafische Ausgabe (Display/Bildschirm) um das Programm anzeigen lassen zu können. Die Anzeige kann gemäß der benutzerdefinierten Auflösung durch Ziehen des Fensters beliebig angepasst werden.
	\item[] \textit{Hardware Schnittstellen:}\\
	Die fertige Software wird in einer Java-Umgebung dargestellt. Daher wird gefordert, dass die Hardware mit der Java-Umgebung ab der in den nicht-funktionalen Anforderungen genannten Version 1.6 kompatibel sein muss, damit die Software reibungslos funktionieren kann.
	\item[] \textit{Software Schnittstellen:}
	\begin{itemize}
	\item[JGraphX:] Eine Java Swing Visualisierungsbibliothek von mxGraph. 
	\item[ISGCI:] Eine bereits umgesetzte Graphenbibliothek, welche viele Informationen über verschiedene Graphklassen beinhaltet (wie Sub-/Superklassen) und ausführliche Informationen derer Eigenschaften.
	\item[JGraphT:] Eine freie Java Klassenbibliothek, die mathematische graphentheoretische Ziele und Algorithmen unterstützt. Diese läuft allerdings nur auf der Java 2 Plattform, welche mindestens JDK 1.6 voraussetzt.
 	\end{itemize}
	Die Software wird eine Schnittstelle zwischen diesen %beiden 
Bibliotheken bilden, indem die Graphzeichnungsmöglichkeiten von JGraphX genutzt werden. Dementsprechend werden die aktuellen Funktionen von ISGCI, die bis dato mit JGraphT umgesetzt sind, auf JGraphX übertragen, indem eine Schnittstelle zwischen JGraphT und JGraphX umgesetzt wird und deren Funktionalität dann von ISGCI auf JGraphX über diese Schnittstelle übertragen wird. 
	
	JGraphT verwendet Maven um den Build Process zur Kompilierung der Java-Klassen und um eine gute Dokumentation zu gewährleisten. Insbesondere führt Maven JUnit-Tests aus, um eben die Kontrolle über die Implementierung zu behalten. Dazu müssen allerdings entsprechende Tests vom Entwickler definiert werden.
	\end{itemize}
	
%	\subsection{Design Constraints (ja das englisch war so vorgegeben und darf gerne geändert werden :-)} %3.4
%	\{Design constraints -- required implementation language, database integrity, limits on usage of resources such as memory, and others.
%	\}
	
	%\subsection{Anforderungen an Performance} %3.5
		
		% KOMMENTAR Keine Änderungen an Performance, Laufzeiten?! - einzige änderung: Graphische Darstellung --> nonfunctional requirements %
		
	%	\{schnelle Laufzeit, wie aktuelles System, damit Änderungen sofort sichtbar sind $\Longrightarrow$ evtl. auch in nicht-funktionalen Anforderungsbereich Punkt 3.2 verschieben!\}
	
	\subsection{Qualitätsanforderungen} %3.6 / Software System Attributes
		\begin{itemize}	
		\item \textbf{Benutzerfreundlichkeit}\\
		Das bestehende System wird vor Allem erweitert um die Benutzerfreundlichkeit zu steigern. Zwar besteht die Zielgruppe für die Software aus technisch interessierten und fähigen Leuten, dennoch soll die Software auch für Laien leicht durchschaubar und selbsterklärend sein. Intuitiv angeordnete Reiter in der Menüleiste, sowie ein interaktives Kontextmenü sollen dazu beitragen, dass benötigte Funktionen von Benutzern leicht gefunden werden können. Des Weiteren sorgen Funktionen wie Grab\&Pull oder Zoom \textit{(vgl. Funktionale Anforderungen)} für einen interaktiven Umgang mit dem System. Um die Übersichtlichkeit über komplexere Graphen zu gewährleisten, gibt es zwei Varianten des Expanding/Collapsing, mit denen man einen komplizierten Graphen in einen überschaubaren Teilgraphen herunterbrechen kann \textit{(vgl. Funktionale Anforderungen 08)}.
		\item \textbf{Integrität}\\
		Alle Datensätze werden zentral auf einem Server gehalten. Die Software bietet ausschließlich lesende Funktionen an. Somit ist die Datenbank durch negative Manipulationen geschützt.
		\item \textbf{Flexibilität}\\
		Nutzer können ihren Wunschgraphen zeichnen lassen, diesen auf verschiedene Arten manipulieren (z.B.: neu anordnen, reduzieren, Knoten-Hierarchien anzeigen lassen) und den entstandenen Graphen exportieren. Dazu lassen sich zu jedem Graph-Knoten (Grapheklasse) Informationen aus der Online-Datenbank einsehen. 
		\item \textbf{Portabilität}\\
		ISGCI ist eine Java Anwendung. Das bedeutet, dass sie plattformunabhängig ausgeführt werden kann, solange eine Java-Laufzeit-Umgebung installiert ist (als Standard vorausgesetzt).
		Die Online-Datenbank ist unabhängig von unserem System. 
		\item \textbf{Wartbarkeit}\\
		Dadurch, dass die Datenbank des Programmes online, zentral gelagert wird, lassen sich Datensätze leicht und ohne Updates im Programm manipulieren oder hinzufügen.
		\end{itemize}
		
%	\subsection{Sonstige Anforderungen} %3.7
%	Other -- database, operations, site adapting, and so on.
%	\subsection{Ergänzende Kommentare} % 3.8
\end{document}
