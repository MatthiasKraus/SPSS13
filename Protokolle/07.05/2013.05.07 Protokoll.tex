\documentclass{scrartcl}

\usepackage[en,break]{ukon-infie}

\Names{L. Kr\"amer}
\Term{SS 13}
\Lecture[SWP]{Softwareprojekt}
\date{10.05.2013}
\title{Protokoll 2. 'Weekly'-Meeting Dienstag 07.05. 8:15}
\def\oder{\vee}
\def\und{\wedge}

\begin{document}
\maketitle
\section{Ort und Zeit}
V-Pool (V304), 8:15-9:00 \\ Ausweichtermin für das wöchentliche Treffen, wegen Fronleichnam
\section{Anwesenheit}
Leonard Krämer, Matthias Kraus, Rebecca Kehlbeck, Fabian Vollmer, Matthias Miller, Philipp Hilpert \\ Entschuldigt: Stephan Heidinger, Nicolas Siebeck
\section{Agenda}
\begin{itemize}
\item Abgeschlossene Tasks
\item Planung nächste Wochen
\item Neue Tasks
\item Ideen Fragen und sonstige TO DO's
\end{itemize}

\section{Abgeschlossene Tasks}
\paragraph{Prototyp 1} Schöner Mock-up von Philipp Hilpert zeigt, dass wir bei einigen Vermutungen daneben lagen und wir ein paar Requirements ändern müssen. Der Prototyp wird nochmal überarbeitet und an den CEO weitergeleitet. Außerdem zeigen wir den Prototyp nach der Projektplanpräsentation.
\paragraph{SRS Planung} Zusammen mit Matthias Kraus erledigt, Aufgaben sind verteilt und werden bearbeitet. Das Inhaltsverzeichnis steht schon fest, darauf wird aufgebaut.
\section{Planung nächste Wochen}
Der Arbeitsaufwand wurde vom Kunden auf 6 Wochen geschätzt, darum versuchen wir so bald wie möglich mit der Implementation zu beginnen
\section{Tasks}
\paragraph{Protokoll Freitag 26.04. fertig machen} Das Protokoll muss überarbeitet werden, damit es als PDF (o.ä.) exportierbar ist. Nico Siebeck übernimmt den Task.
\paragraph{SRS Erstellen} Milestone Task, Spezifikationsmanager hat die Leitung. Da die Aufgaben schon verteilt sind gibt es hier nicht viel zu sagen, jeweils in 2er Teams werden einzelne Punkte des SRS geschrieben und dann zusammengefügt. Das ganze passiert auf GitHub für Diffs.

\paragraph{Projektplanpräsentation} Präsentation ist mit Google Drive erstellt worden, sie wird gemeinsam mit dem PM am Nachmittag gehalten, vorher wird noch einmal geübt.
\paragraph{GIT Präsentation} Präsentation mit Thorsten Sauter, Rebecca übernimmt die Präsentation am 21.05.
\section{Ideen Fragen und sonstige TO DO's}
Philipp Hilpert: Termine für Meetings müssen immer 2 Tage im Voraus angekündigt werden
\end{document}