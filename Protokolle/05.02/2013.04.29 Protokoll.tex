\documentclass{scrartcl}

\usepackage[en,break]{ukon-infie}

\Names{L. Kr\"amer}
\Term{SS 13}
\Lecture[SWP]{Softwareprojekt}
\date{30.04.2013}
\title{Protokoll 3. Treffen Montag 29.04. 13:00}
\def\oder{\vee}
\def\und{\wedge}

\begin{document}
\maketitle
\section{Ort}
V-Pool (V304), 17:00-17:30
\section{Anwesenheit}
Leonard Krämer, Matthias Kraus, Rebecca Kehlbeck, Fabian Vollmer, Nicolas Siebeck, Matthias Miller, Philipp Hilpert, Staphan Heidinger
\section{Agenda}
\begin{itemize}
\item Vorstellung Stephan
\item Abgeschlossene Tasks
\item Buidprobleme
\item Planung nächste Wochen
\item Neue Tasks
\item Ideen Fragen und sonstige TO DO's
\end{itemize}
\section{Vorstellung Stephan}
Stephan will alles was wir an Material produzieren und ist so lange eine Hilfe für uns, bis seine Stunden aufgebraucht sind.
Er steht gerne für Rückfragen und hilfreiche Informationen zur Verfügung \\
Stephan kriegt in Zukunft das gesamte Material und wird Teilnehmer aller AgreeDo Meetings
\section{Abgeschlossene Tasks}
\begin{itemize}
\item Projektplan erstellen \\ Der Projektplan ist fertig, es wird maximal noch an Formulierungen gefeilt.
\item Git-Präsentation \\ Alle haben sich in Git eingearbeitet und können jetzt (rudimentär) damit umgehen
\item SRS Planung \\ Ist abgeschlossen und die Aufgaben werden bearbeitet
\item Fragen Concall \\ Sind erarbeitet, Herr de Ridder wird morgen 9:00 angerufen (Leonard Krämer und Matthias Kraus) Das ergebnis wird über Drive weitergeleitet
\end{itemize}
\section{Buildprobleme}
Offenbar hatte Stephan einige Probleme ISGCI zu bauen, mittlerweile hat er das behoben und Fabian Vollmer in die Lösung eingewiesen
\section{Planung nächste Wochen}
Der Arbeitsaufwand wurde vom Kunden auf 6 Wochen geschätzt, darum versuchen wir so bald wie möglich mit der Implementation zu beginnen
\section{Tasks}
\paragraph{Protokoll Freitag 26.04. fertig machen} Das Protokoll muss überarbeitet werden, damit es als PDF (o.ä.) exportierbar ist
\paragraph{SRS Erstellen} Milestone Task, Spezifikationsmanager hat die Leitung
\paragraph{Library Experts}
\subparagraph{ISGCI} Übernimmt Fabian Vollmer, er wird sich intensiv einarbeiten
\subparagraph{JgraphX} Übernimmt Matthias Miller: JGraphX - JGraphT welche Methoden Schnittstelle von alten JGraph müssen implementiert und verstanden werden...

Schauen wie die Schnittstellen funktionieren, herumspielen mit JgraphX, z.B. Haus vom Nikolaus zeichnen
\paragraph{Software}
Dropbox wird abgesetzt, führt nur zu doppelter Datenhaltung \\
Über Maven, Eclipse und ANT gibts noch Präsentationen in der Vorlesung, eventuell müssen wir vorarbeiten, wenn das zu langsam geht.
\paragraph{Projektplanpräsentation} Präsentationsmanager erstellt die Präsentation mit Google Drive, sie wird gemeinsam mit dem PM am 7.5. gehalten.
\paragraph{GIT Präsentation} Präsentation mit Thorsten Sauter, abklären ob Rebecca die Präsentation übernehmen kann.
\section{Ideen Fragen und sonstige TO DO's}
N/A
\end{document}