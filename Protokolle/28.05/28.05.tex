\documentclass{scrartcl}

\usepackage[en,break]{ukon-infie}

\Names{L. Kr\"amer}
\Term{SS 13}
\Lecture[SWP]{Softwareprojekt}
\date{28.05.2013}
\title{Protokoll 5. 'Weekly'-Meeting Dienstag 28.05. 11:35}
\def\oder{\vee}
\def\und{\wedge}

\begin{document}
\maketitle
\section{Ort und Zeit}
V-Pool (V304), 13:35-14:15 \\
\section{Anwesenheit}
Leonard Krämer, Rebecca Kehlbeck, Stephan Heidinger, Matthias Miller, Philipp Hilpert, Nicolas Siebeck \\ Entschuldigt: Matthias Kraus, Fabian Vollmer
\section{Agenda}
\begin{itemize}
\item Git
\item Abgeschlossene Tasks
\item Planung nächste Wochen
\item Neue Tasks
\item Ideen Fragen und sonstige TO DO's
\end{itemize}
\section{Git} Es kam zu Problemen mit Git, da alle wild commited haben. Darum werden wir ab sofort alle das Projekt SPSS13 forken und Rebecca Kehlbeck und Philipp Hilpert managen die 'Pull'-Requests \\ Zusätzlich wird das Git-Repository aufgeräumt.
\section{Abgeschlossene Tasks}
\paragraph{SDD} 'Bestehendes System' ist fertig beschrieben, es kommt noch eine Rechtschreibkorrektur.
\section{ Planung nächste Wochen}
Das nächste 'Weekly'-Treffen wird am 06.06. um 17:00 stattfinden
\paragraph{$\alpha$-Version} Wir konnten diesen Milestone nicht halten, da das SDD aufgrund der Fehlannahme mit Jgraph noch nicht den nötigen Stand hat. Der CEO wird informiert.
\paragraph{Coding-Day} Um den Rückstand aufzuholen machen wir einen 'Coding-Day', also ein Bootcamp um die Alphaversion zu erstellen.
\section{Tasks}
\paragraph{SDD} Beschreiben der Vorgeschlagenen Architektur: \\ GUI: Philipp Hilpert, Fabian Vollmer, Nicolas Siebeck\\ 
Graphzeichnen: Matthias Miller, Leonard Krämer

\paragraph{GIT Präsentation} Präsentation mit Thorsten Sauter, Rebecca übernimmt die Präsentation am 28.05.
\paragraph{Coding-Day} Um den Rückstand aufzuholen machen wir einen 'Coding-Day', also ein Bootcamp um die Alphaversion zu erstellen.
\section{Ideen Fragen und sonstige TO DO's}
\paragraph{neues Design für zusätzliche Informationen} Wir haben das neue Design besprochen und werden dafür Mockups erstellen um mit Herrn de Ridder darauf einzugehen. Es wird ein zusätzliches Feld am Bildrand geben, um Informationen, wie alternative Namen, oder Definitionen der Graphklassen anzuzeigen. Die Mockups sind Fertig und warten auf einen Review.
\end{document}